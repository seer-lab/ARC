\section{Future Work}
\label{sec:future_work}

As stated above, ARC isn't quick.  We plan to optimize issues around speed:
Immediate evaluation of potential fixes -- instead of at the end of a
generation, memory caching and mutation creation on demand, instead of
exhaustively. ConTest is the largest drain on resources in terms of ARC. There
has been some work on exploring optimizations in terms of the configuration of
ConTest, which we can take advantage of as we currently use the default
configuration~\cite{KLVU10}.

Targeting the variables, functions and classes on the error path is an
important feature  we would like to add. It has shown great success in helping
to isolate and automatically fix bugs in single-threaded programs~\cite{FNWG09,
WNLF09, NWLF09, WFGN10, GNFW11, LDFW12}. It should be straight forward to
integrate a static analysis tool~\cite{HP04, NPSG09, NWLF09} like Chord.  We
seed the first generation with mutants from the target area and grow the search
from there.

We plan to experiment with new operators. Potential additions include splitting
or merging synchronization blocks and adding new locks. Through experimentation
we plan to optimize the existing set of mutation operators to maximize their
capabilities.  Many new concurrency structures were introduced in Java 5.
Expanding ARC's operators to deal with new anti-patterns~\cite{BJ09, FKLV12,
BCD06} gives ARC the ability to fix additional types of bugs. An intriguing
possibility is adding crossover, changing ARC into a genetic or gene expression
algorithm.
