\section{Experiments}
\label{sec:experiments}

% Due to the ARCs heuristic nature it is entirely possible for a program to
% perform better or worse then it did originally. We are interested in both ARCs
% ability to find fixes and ARCs effect on performance of solutions found.

\subsection{Experimental Setup}
\label{sec:experimental_setup}

We evaluate ARC's ability to repair concurrency bugs by conducing an experiment
to gather preliminary results. The 8 buggy programs we chose come from a
subset of the IBM Concurrency Benchmark~\cite{EHSU06}. Our selection criteria
was to pick a subset of programs containing bugs that ARC has the
capability of fixing based on the set of mutation operators. For example, we
ignored programs that used some of the newer concurrency structures from Java 5. We
also included some programs that we identified that ARC could not fix as a
sanity check.

Unfortunately, none of the programs contained the JUnit test suite that ARC requires.
We manually created JUnit test suites without altering the programs and their
bugs's behaviour\footnote{One incorrect program was fixed so that it properly
exhibited its concurrency bug.}. Each test suite consists of three unit tests
each with a different \textit{level} of concurrency\footnote{IBM Concurrency Benchmark
programs have an argument that controls the number of threads that are used in
execution.}. Source code details on the programs are presented in
Table~\ref{tbl:used_programs}

% TODO Maybe just remove the Bug Pattern (as it would need more explanation)
\begin{table}[h]
\caption{The set of programs used to evaluate ARC. The test suite for each
program is excluded from these values.}
\begin{center}
\begin{tabular}{|l|r|r|l|l|l|}
\hline
\textbf{Program} & \textbf{SLOC} & \textbf{Classes} & \textbf{Bug Type} & \textbf{Can Fix?} & \textbf{Bug Pattern}\\
\hline
account & 165 & 3 & Data Race & Yes & NoLock\\
\hline
accounts & 75 & 2 & Data Race & Yes & NonAtomicAssumedAtomic\\
\hline
airline & 93 & 1 & Data Race & No & Interference\\
\hline
%allocation & 165 & 3 & Data Race & Yes & TwoStageAccess\\
%\hline
%bubble & 246 & 4 & Data Race & No & NonAtomicAssumedAtomic,\newline OrphanedThread\\
%\hline
bubblesort2 & 104 & 2 & Data Race & No & Initialization-Sleep\\
\hline
buffer & 319 & 5 & Data Race & No & NotifiyInsteadOfNotifyAll\\
\hline
%bufwriter & 170 & 5 & Deadlock & Yes & NoLock\\
%\hline
deadlock & 109 & 2 & Deadlock & Yes & Deadlock\\
\hline
lottery & 157 & 2 & Data Race & No & NonAtomicAssumedAtomic,\newline NoLock, BlockingCriticalSection\\
\hline
%mergesort & 281 & 2 & Data Race & No & NonAtomicAssumedAtomic\\
%\hline
pingpong & 143 & 4 & Data Race & Yes & NonAtomicAssumedAtomic\\
\hline
\end{tabular}
\label{tbl:used_programs}
\end{center}
\end{table}

ARC was designed to be flexible in terms of the parameters that can be
configured. Table~\ref{tbl:used_parameters} lists and describes each parameter,
including the values selected for evaluation. Values were selected based on our
experience with ARC. The selected parameters allowed for 10 ConTest executions
to occur to evaluate each individual at each generation. If a potential
solution is found, the validation of that potential solution will execute an
addition 150 ConTest executions (10 ConTest Runs $\times$ Validation Mult. of
15).

% TODO Maybe reduce this to the common ES parameters?
\begin{table}%[h]
\caption{The set of parameters that ARC uses along with their descriptions and
used values for the experimentations.}
\begin{center}
\lstset{basicstyle=\scriptsize}
\begin{tabular}{|l|l|r|}
\hline
\textbf{Parameter} & \textbf{Description} & \textbf{Value}\\
\hline
Project Test MB & The amount of memory allocated for the testing & 2000\\
\hline
ConTest Runs & Test suite executions per gen. per member & 10\\
\hline
Validation Mult. & Multiplier on ConTest runs when validating the functionality & 15\\
\hline
Timeout Mult. & Time multiplier for ConTest before timeout & 20\\
\hline
Evolution Gen & Maximum number of generations in fixing phase & 30\\
\hline
Evolution Population & Population size for the evolutionary strategy & 30\\
\hline
Replace Lowest \% & Lowest $n$\% of population replaced & 10\\
\hline
Replace With Best \% & Replace underperfomer with best individual $n$\% of the time & 75\\
\hline
Replace min turns & Minimum time underperforming & 3\\
\hline
Replace Interval & Every $n$ generations, underperformers are replaced & 5\\
\hline
Ranking Window & Size of sliding window for operator weighting & 5\\
\hline
Success Weight & The weighting applied for successful executions & 100\\
\hline
Timeout Weight & The weighting applied for timeout executions & 50\\
\hline
Improv. Window & Size of window for convergence check & 10\\
\hline
Avg. Fit. Delta & Minimum average fitness improvement required & 0.01\\
\hline
Best Fit. Delta & Minimum best fitness improvement required & 1\\
\hline
\end{tabular}
\label{tbl:used_parameters}
\end{center}
\end{table}

\subsection{Experimental Results}
\label{sec:experimental_results}

Each program was run through ARC a total of 5 times using the parameters
described in Table~\ref{tbl:used_parameters} and Programs in
Table~\ref{tbl:used_programs}. Results for each program is summarized in
Table~\ref{tbl:summary_results}.

\begin{table}%[h]
\caption{Summary of the results of running the programs
(from Table~\ref{tbl:used_programs}) through ARC 5 times.}
\begin{center}
\lstset{basicstyle=\scriptsize}
\begin{tabular}{|l|l|l|}
\hline
\textbf{Program} & \textbf{Gen. Fix Found} & \textbf{Time Taken (h:m:s)}\\
\hline
account & 4, 15, 4, 1, 1 & 0:19:53, 1:08:30, 0:19:54, 0:06:39, 0:06:29\\
\hline
accounts & 1, 1, 1, 1, 1 & 2:02:57, 2:08:11, 2:05:31, 2:10:18, 2:36:24\\
\hline
airline & --, --, --, --, -- & 3:58:07, 3:56:19, 3:49:34, 3:53:14, 3:57:51\\
\hline
%allocation & TO, TO, TO, 3, TO & --, --, --, 22:39:6, --  \\
%\hline
%bubble & 28, & 5:20:33, \\
%\hline
bubblesort2 & 2, 2, 3, 2, 2 & 3:15:06, 5:11:04, 5:53:14, 6:13:16, 7:05:32 \\
\hline
buffer & --, --, --, --, -- & 4:43:50, 4:45:12, 5:07:03, 4:58:27, 5:02:30\\
\hline
%bufwriter & TO, TO, TO, TO, TO & --, --, --, --, --\\
%\hline
deadlock & 1, 1, 1, 1, 1 & 0:06:45, 0:05:43, 0:06:45, 0:05:40, 0:07:51\\
\hline
lottery & 3, 4, 2, 1, 2 & 2:19:18, 2:43:18, 1:54:00, 1:05:27, 1:28:39\\
\hline
%mergesort & & \\
%\hline
pingpong & 1, 1, 1, 1, 1 & 0:39:57, 0:12:47, 0:46:21, 0:46:16, 0:42:02\\
\hline
\end{tabular}
\label{tbl:summary_results}
\end{center}
\end{table}

As we can see from the results, the time taken to find a fix range from
about 6 minutes to almost a whole day. ARC is currently not an efficient process
as it is under-optimized in many aspects. The main bottle-neck in terms of time
and resources within ARC is the repeated number of ConTest executions. ConTest
instrumented a program with random noise that negatively effects the performance 
of the execution.  The \textit{Timeout Multiplier} in Table~\ref{tbl:used_parameters} allows ARC to wait up to 20 times the normal execution time for the program under test to complete due to the noise. 

% TODO Add discussion of ARCs fix finding capabilities. Does it find fixes for
% the programs we figured ARC should be able too, did it find fixes for those
% we didn't thing it could (how?). Any comment on the generation fixes were
% found on?

% The state spaces of the found fixes, we would expect the larger state space
% programs to have their fixes found later. This does not seem to be the case.
% allocation 76
% deadlock - 89
% accounts - 102
% bubblesort2 - 103
% pingpong - 123
% account - 163
% bubble - 178
% lottery - 188
