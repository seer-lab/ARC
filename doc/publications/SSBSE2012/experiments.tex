\section{Experiments}
\label{sec:experiments}

%Due to the ARCs heuristic nature it is entirely possible for a program to
%perform better or worse then it did originally. We are interested in both ARCs
%ability to find fixes and ARCs effect on performance of solutions found.

ARC is not a quick process. Executing ConTest (population $\times$ generations
$\times$ \# runs $\times$ ConTest run time) many times alone can take
significant resources. Execution times on programs in the experimental section
have ranged between roughly 5 minutes and 4 hours.

\subsection{Experimental Setup}
\label{sec:experimental_setup}

ARC is evaluated against selected programs from the IBM Concurrency
Benchmark~\cite{EHSU06}. This set of programs is rather small yet demonstrate a
variety of concurrent bug types. Unfortunately, none of the programs contain
JUnit test suite that ARC requires. We manually created JUnit test suites
without altering the programs and their bugs's behaviour\footnote{One incorrect
program was fixed so that it properly exhibited its concurrency bug.}. Details
on the programs are presented in Table~\ref{tbl:used_programs}.
% TODO Is the footnote still true? (CC: David)

% TODO Maybe just remove the Bug Pattern (as it would need more explanation)
\begin{table}[h]
\caption{The set of programs used to evaluate ARC. The test suite for each
program is excluded from these values.}
\begin{center}
\begin{tabular}{|l|r|r|l|l|}
\hline
\textbf{Program} & \textbf{SLOC} & \textbf{Classes} & \textbf{Bug Type} & \textbf{Bug Pattern}\\
\hline
account & 165 & 3 & Data Race & NoLock\\
\hline
accounts & 75 & 2 & Data Race & NonAtomicAssumedAtomic\\
\hline
airline & 93 & 1 & Data Race & Interference\\
\hline
allocation & 165 & 3 & Data Race & TwoStageAccess\\
\hline
bubble & 246 & 4 & Data Race & NonAtomicAssumedAtomic,\newline OrphanedThread\\
\hline
bubblesort2 & 104 & 2 & Data Race & Initialization-Sleep\\
\hline
buffer & 319 & 5 & Data Race & NotifiyInsteadOfNotifyAll\\
\hline
bufwriter & 170 & 5 & Deadlock & NoLock\\
\hline
deadlock & 109 & 2 & Deadlock & Deadlock\\
\hline
lottery & 157 & 2 & Data Race & NonAtomicAssumedAtomic,\newline NoLock, BlockingCriticalSection\\
\hline
mergesort & 281 & 2 & Data Race & NonAtomicAssumedAtomic\\
\hline
pingpong & 143 & 4 & Data Race & Null-Pointer Exception\\
\hline
\end{tabular}
\label{tbl:used_programs}
\end{center}
\end{table}

ARC was designed to be flexible in terms of the parameters that can be
configured. Table~\ref{tbl:used_parameters} lists and describes each parameter,
including the values selected for evaluation. Values were selected based on our
experience with ARC.

% TODO Maybe reduce this to the common ES parameters?
\begin{table}%[h]
\caption{The set of parameters that ARC uses along with their descriptions and
used values for the experimentations.}
\begin{center}
\lstset{basicstyle=\scriptsize}
\begin{tabular}{|l|l|r|}
\hline
\textbf{Parameter} & \textbf{Description} & \textbf{Value}\\
\hline
Project Test MB & The amount of memory allocated for the testing & 2000\\
\hline
ConTest Runs & Test suite executions per gen. per member & 10\\
\hline
Validation Mult. & Multiplier on ConTest runs when validating the functionality & 15\\
\hline
Timeout Mult. & Time multiplier for ConTest before timeout & 15\\
\hline
Evolut. Gen. & Maximum number of generations in fixing phase & 30\\
\hline
Evolution Population & Population size for the evolutionary strategy & 30\\
\hline
Replace Lowest \% & Lowest $n$\% of population replaced & 10\\
\hline
Replace With Best \% & Replace underperfomer with best individual $n$\% of the time & 75\\
\hline
Replace min turns & Minimum time underperforming & 3\\
\hline
Replace Interval & Every $n$ generations, underperformers are replaced & 5\\
\hline
Ranking Window & Size of sliding window for operator weighting & 5\\
\hline
Success Weight & The weighting applied for successful executions & 100\\
\hline
Timeout Weight & The weighting applied for timeout executions & 50\\
\hline
Improv. Window & Size of window for convergence check & 10\\
\hline
Avg. Fit. Delta & Minimum average fitness improvement required & 0.01\\
\hline
Best Fit. Delta & Minimum best fitness improvement required & 1\\
\hline
\end{tabular}
\label{tbl:used_parameters}
\end{center}
\end{table}

\subsection{Experimental Results}
\label{sec:experimental_results}

Each program was run through ARC a total of 5 times using the parameters
described in Table~\ref{tbl:used_parameters}. The results of each program is
summarized in Table~\ref{tbl:summary_results}.

\begin{table}%[h]
\caption{Summary of the results of running the programs
(Table~\ref{tbl:used_programs}) through ARC 5 times.}
\begin{center}
\lstset{basicstyle=\scriptsize}
\begin{tabular}{|l|l|l|}
\hline
\textbf{Program} & \textbf{Gen. Fix Found} & \textbf{Time Taken (h:m:s)}\\
\hline
account & 4, 15, 4, 1, 1 & 0:19:53, 1:08:30, 0:19:54, 0:06:39, 0:06:29\\
\hline
accounts & 1, 1, 1, 1, 1 & 2:02:57, 2:08:11, 2:05:31, 2:10:18, 2:36:24\\
\hline
airline & --, --, --, --, -- & 3:58:07, 3:56:19, 3:49:34, 3:53:14, 3:57:51\\
\hline
allocation & TO, TO, TO, 3, TO & --, --, --, 22h 39m 6s, --  \\
\hline
bubble & 28, & 5:20:33, \\
\hline
bubblesort2 & 2, 2, 3, 2, -- & 3:15:06, 5:11:04, 5:53:14, 6:13:16, -- \\
\hline
buffer & --, --, --, --, -- & 4:43:50, 4:45:12, 5:07:03, 4:58:27, 5:02:30\\
\hline
bufwriter &  & \\
\hline
deadlock & 1, 1, 1, 1, 1 & 0:06:45, 0:05:43, 0:06:45, 0:05:40, 0:07:51\\
\hline
lottery & 3, 4, 2, 1, TO & 2:19:18, 2:43:18, 1:54:00, 1:05:27, --\\
\hline
mergesort & & \\
\hline
pingpong & 1, 1, 1, 1, 1 & 0:39:57, 0:12:47, 0:46:21, 0:46:16, 0:42:02\\
\hline
\end{tabular}
\label{tbl:summary_results}
\end{center}
\end{table}
