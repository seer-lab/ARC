\section{Evaluation}
\label{sec:experiments}

We will now describe the experimental setup and the results obtained from our evaluation of ARC.

% Due to the ARCs heuristic nature it is entirely possible for a program to
% perform better or worse then it did originally. We are interested in both ARCs
% ability to find fixes and ARCs effect on performance of solutions found.

\subsection{Experimental Setup}
\label{sec:experimental_setup}

In order to evaluate ARC's ability to repair concurrency bugs we selected 8 buggy programs from a subset of the IBM Concurrency Benchmark~\cite{EHSU06}. Our selection
criteria was to pick six programs containing bugs that ARC has the
capability of fixing. We also included two programs that we believed ARC could not fix as a sanity check.

Unfortunately, none of the programs contained the JUnit test suite that ARC
requires. We manually created JUnit test suites without altering the programs
and their bugs's behaviour\footnote{One incorrect program was fixed so that it
properly exhibited its concurrency bug.}. Each test suite consists of three
unit tests each with a different \textit{level} of concurrency\footnote{IBM
Concurrency Benchmark programs have an argument that controls the number of
threads that are used in execution.}. Source code details on the programs are
presented in Table~\ref{tbl:used_programs}

% TODO Maybe just remove the Bug Pattern (as it would need more explanation)
\begin{table}[h]
\caption{The set of programs used to evaluate ARC. The test suite for each
program is excluded from these values.}
\begin{center}
\begin{tabular}{|l|r|r|l|l|p{4.5cm}|}
\hline
\textbf{Program} & \textbf{SLOC} & \textbf{Classes} & \textbf{Bug Type} & \textbf{Can Fix?} & \textbf{Bug Pattern}\\
\hline
account & 165 & 3 & Data Race & Yes & NoLock\\
\hline
accounts & 75 & 2 & Data Race & Yes & NonAtomicAssumedAtomic\\
\hline
airline & 93 & 1 & Data Race & No & Interference\\
\hline
%allocation & 165 & 3 & Data Race & Yes & TwoStageAccess\\
%\hline
%bubble & 246 & 4 & Data Race & No & NonAtomicAssumedAtomic,\newline OrphanedThread\\
%\hline
bubblesort2 & 104 & 2 & Data Race & Yes & Initialization-Sleep\\
\hline
buffer & 319 & 5 & Data Race & No & NotifiyInsteadOfNotifyAll\\
\hline
%bufwriter & 170 & 5 & Deadlock & Yes & NoLock\\
%\hline
deadlock & 109 & 2 & Deadlock & Yes & Deadlock\\
\hline
lottery & 157 & 2 & Data Race & Yes & NonAtomicAssumedAtomic,\newline NoLock, BlockingCriticalSection\\
\hline
%mergesort & 281 & 2 & Data Race & No & NonAtomicAssumedAtomic\\
%\hline
pingpong & 143 & 4 & Data Race & Yes & NonAtomicAssumedAtomic\\
\hline
\end{tabular}
\label{tbl:used_programs}
\end{center}
\end{table}

ARC was designed to be flexible in terms of the underlying evolutionary strategy parameters. Table~\ref{tbl:used_parameters} lists each parameter and its value. Values were selected based on our
experience with ARC. The selected parameters allowed for 10 ConTest executions
to occur to evaluate each individual at each generation. If a potential
solution is found, the validation of that potential solution will execute an
addition 150 ConTest executions (10 ConTest Runs $\times$ Validation Mult. of
15).

% TODO Maybe reduce this to the common ES parameters?
\begin{table}[t!]
\caption{The set of parameters that ARC uses along with their descriptions and
 values.}
\begin{center}
\lstset{basicstyle=\scriptsize}
\begin{tabular}{|l|p{7.5cm}|r|}
\hline
\textbf{Parameter} & \textbf{Description} & \textbf{Value}\\
\hline
Project Test MB & The amount of memory allocated for the testing & 2000\\
\hline
ConTest Runs & Test suite executions per gen. per member & 10\\
\hline
Validation Mult. & Multiplier on ConTest runs when validating the functionality & 15\\
\hline
Timeout Mult. & Time multiplier for ConTest before timeout & 20\\
\hline
Evolution Gen & Maximum number of generations in fixing phase & 30\\
\hline
Evolution Population & Population size for the evolutionary strategy & 30\\
\hline
Replace Lowest \% & Lowest $n$\% of population replaced & 10\\
\hline
Replace With Best \% & Replace underperfomer with best individual $n$\% of the time & 75\\
\hline
Replace min turns & Minimum time underperforming & 3\\
\hline
Replace Interval & Every $n$ generations, underperformers are replaced & 5\\
\hline
Ranking Window & Size of sliding window for operator weighting & 5\\
\hline
Success Weight & The weighting applied for successful executions & 100\\
\hline
Timeout Weight & The weighting applied for timeout executions & 50\\
\hline
Improv. Window & Size of window for convergence check & 10\\
\hline
Avg. Fit. Delta & Minimum average fitness improvement required & 0.01\\
\hline
Best Fit. Delta & Minimum best fitness improvement required & 1\\
\hline
\end{tabular}
\label{tbl:used_parameters}
\end{center}
\end{table}

\subsection{Experimental Results}
\label{sec:experimental_results}

Each program was run through ARC a total of 5 times using the parameters
described in Table~\ref{tbl:used_parameters} and the programs in
Table~\ref{tbl:used_programs}. Results for all 5 runs of each program are summarized in
Table~\ref{tbl:summary_results}.

\begin{table}[t!]
\caption{Summary of the results of running the programs through ARC 5 times.}
\begin{center}
\lstset{basicstyle=\scriptsize}
\begin{tabular}{|l|l|l|}
\hline
\textbf{Program} & \textbf{Gen. Fix Found} & \textbf{Time Taken (h:m:s)}\\
\hline
account & 4, 15, 4, 1, 1 & 0:19:53, 1:08:30, 0:19:54, 0:06:39, 0:06:29\\
\hline
accounts & 1, 1, 1, 1, 1 & 2:02:57, 2:08:11, 2:05:31, 2:10:18, 2:36:24\\
\hline
airline & --, --, --, --, -- & 3:58:07, 3:56:19, 3:49:34, 3:53:14, 3:57:51\\
\hline
%allocation & TO, TO, TO, 3, TO & --, --, --, 22:39:6, --  \\
%\hline
%bubble & 28, & 5:20:33, \\
%\hline
bubblesort2 & 2, 2, 3, 2, 2 & 3:15:06, 5:11:04, 5:53:14, 6:13:16, 7:05:32 \\
\hline
buffer & --, --, --, --, -- & 4:43:50, 4:45:12, 5:07:03, 4:58:27, 5:02:30\\
\hline
%bufwriter & TO, TO, TO, TO, TO & --, --, --, --, --\\
%\hline
deadlock & 1, 1, 1, 1, 1 & 0:06:45, 0:05:43, 0:06:45, 0:05:40, 0:07:51\\
\hline
lottery & 3, 4, 2, 1, 2 & 2:19:18, 2:43:18, 1:54:00, 1:05:27, 1:28:39\\
\hline
%mergesort & & \\
%\hline
pingpong & 1, 1, 1, 1, 1 & 0:39:57, 0:12:47, 0:46:21, 0:46:16, 0:42:02\\
\hline
\end{tabular}
\label{tbl:summary_results}
\end{center}
\end{table}

As we can see from the results, the time taken to repair the concurrency programs range from about 6
minutes to almost 3 hours. ARC is currently not an efficient process as it is
under-optimized in many aspects. The main bottle-neck in terms of time and
resources within ARC is the repeated number of ConTest executions. ConTest
instrumented a program with random noise, causing threads to delay and thus negatively effecting the
execution time.  The \textit{Timeout Multiplier} in
Table~\ref{tbl:used_parameters} allows ARC to wait up to 20 times the instrumented
execution time for the program under test to complete due to the noise.

% Results of inspecting the programs and their fixes
% Lottery all have the same solution (correct solution) (ARC can fix)
% Bubblesort2 all have the same solution (correct solution, makes it sequential) (ARC can fix)
% Pingpong all essentially the same solution (some synch the method itself, some synch around the method call) (ARC can fix)
% Deadlock all essentially the same solution (a calls b, synch a or b) (ARC can fix)
% Accounts all the same solution (correct solution) (ARC can fix)
% Account all the same solution (makes the program sequential, not the optimal solution [limitation of ARC, ConTest does not identify all shared variable name changes]) (ARC can fix)
% Airline no fixes (ARC cannot fix this, logic problems that involves moving code around)
% Buffer no fixes (ARC cannot fix this, as it is a notify vs. notifyall bug)

To better explain our results we will now provide details of the specific fixes identified by ARC. Overall ARC was able to fix all 6 of the programs we believed ARC should be capable of fixing. The fix for \textit{accounts} is achieved by synchronizing a method that handles where the data race occurs.  The \textit{pingpong} and
\textit{deadlock} program have similar bug patterns and fixes.  In each program
a \texttt{methodA} calls a \texttt{methodB}.  There is a data race in
\texttt{methodB} so synchronizing either \texttt{methodA} or \texttt{methodB}
fixes the program.  In these programs this is a valid solution as there are no
other methods that call \texttt{methodB}.  ARC fixes \textit{bubblesort2}, \textit{lottery} and \textit{account} by synchronizing the \texttt{run()} method of the threads.  Unfortunately, this has the side-effect of making the
programs sequential.  After further inspected these solution we found that only the \texttt{account} program's fix is sub-optimal.  Only one of the
methods called from \texttt{account}'s  \texttt{run()} method  needs to be
synchronized, unfortunately ConTest could not properly identify the variable
argument. ARC was unable to use the proper mutation operator and ends synchronizing the whole \texttt{run()} method.

We had two programs for which ARC was unable to find a fix -- these two programs were initially identified as not being fixable by ARC. \textit{buffer} has an unsupported bug pattern of
\textit{notify vs. notifyall}. ARC does not have the the necessary mutation
operator to fix this bug pattern (uses Java 5 concurrency structure). The fix
for \textit{airline} requires the reorganization of code -- which ARC was not
designed to perform.  A number of the fixes are found in the first generation,
as the selected programs are rather small in size as well as state space due to
the limited number of concurrency structures and shared variables.  After
inspecting the fixes of a program, we realized that ARC has reached the same
solution through different mutations. It is the case that different mutations
lead to the same result (i.e., synchronizing on a lock object vs. synchronizing
the method), this is another reason why there are many fixes found in the first
generation  In  some cases there is some variation in the generation in which a
fix is found, this is mostly the result of searching a larger state space.
