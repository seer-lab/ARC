\section{Introduction}
\label{sec:introduction}

As computers and mobile devices now ship with more than one processor, programs
must take advantage of multiple processes as we have approached the end of
the\textit{free-lunch}~\cite{SL05}. Parallelism introduces a new class of bugs
and what makes them difficult to find is that they only occur in a small set of
execution interleavings~\cite{MQB07}. Even when a concurrency bug as been
detected, the repair of such a bug may not be apparent to the developer. It is
often the case that there are multiple threads and data accesses occurring at
the same time in different locations within the program.

We propose ARC (\textbf{A}utomatic \textbf{R}epair of \textbf{C}oncurrency
bugs): An automatic technique to repair deadlocks and data races in parallel
Java programs. Formal specifications, annotations and elaborate test suites are
not required. Only the Java source code and test(s) demonstrating the deadlocks
and/or data races are necessary. Evolutionary strategies (ES) are used to
evolve variants of the program until one is found that fixes the bugs in
question.

The use of search-based software engineering (SBSE)~\cite{Har+10} techniques to
automatically repair bugs is not a novel idea~\cite{FNWG09, WNLF09, NWLF09,
WFGN10, GNFW11, LDFW12}. Our proposed approach adapts the original idea of
automatically fixing \textit{sequential} programs to specifically target
\textit{concurrent} programs.

ARC uses ES to iteratively evolve a buggy program into a version that contains
the proper synchronization that resolves the concurrency bug. To evaluate
whether a program is fixed or now ARC takes advantage of the IBM's
ConTest~\cite{EFN+02} tool, which is used to explore different interleavings of
the program under test.

A common criticism in automatic bug repair techniques is large search space,
not to mention that parallelism introduces thread interleavings on top of this.
ARC innately lessens some of these problems due to a few design choices. First,
we limit the algorithm to only fixing deadlocks and data races. Second, ARC
only targets concurrency mechanisms (\texttt{synchronize} statements are added,
removed, and manipulated). Third, we use a specific set of 11 TXL~\cite{CHP91}
operators based on the ConMAn suite~\cite{BCD06} to mutate the concurrent
program.  Finally, ARC only considers shared variables during the mutation
process. By considering these four design choices ARC operates on a limited set
of operators, and possible places where mutation can occur.

We cover background material related to concurrency and evolutionary strategies
in Sect.~\ref{sec:background}. The motivation for ARC along with an example
problem are presented in Sect.~\ref{sec:motivation}. In
Sect.~\ref{sec:approach} we describe the approach ARC uses to evolve a fix for
a concurrency bug using ES. We evaluate the fixing effectiveness of ARC in
Sect.~\ref{sec:experiments} by conducing an experiment using several programs.
Three challenges that ARC faces are identified in Sect.~\ref{sec:challenges}.
Threats to validity are covered in Sect.~\ref{sec:threats}. We describe the
current ongoing research to add a second phase to ARC that attempts to improve
the performance of the found fixes in Sect.~\ref{sec:ongoing}. In
Sect.~\ref{sec:related_works} we mention related works in the field of
automatic bug repair. Finally, future work and conclusions are covered in
Sect.~\ref{sec:future_work} \& Sect.~\ref{sec:conclusion}.

% In ongoing work we are adding a second phase to the system.  It attempts to
% improve performance by shrinking and removing synchronization blocks. As this
% can introduce data races or deadlocks, any mutant decreasing correctness is
% rejected. This second phase is still in development.  In the rest of the paper
% we concentrate on the first phase, bug fixing.

% To the best of our knowledge there has been no previous work using evolutionary
% strategies to fix bugs in concurrent software. There has been work involving
% the correction of concurrency bugs using self-healing~\cite{LVK08}. From the
% paper, \textit{The healing techniques based on influencing the scheduling do
% not guarantee that a detected problem will really be completely removed, but
% they can decrease the probability of its manifestation.} In contrast ARC is an
% off-line technique that fixes bugs by modifying the source code.

% The main contributions of this paper are:

% \begin{itemize}

% \item An algorithm to create fixes for deadlocks and data races in concurrent Java
% programs. Only the source code and tests demonstrating the bugs are necessary.
% To the best of our knowledge this is the first approach to fix both kinds of
% bugs in Java programs.

% \item Methods to constrain the search space: First, by specifically targeting
% synchronization mechanisms. Second, by using a limited number of TXL operators
% to transform the Java source.  Third, by targeting the variables used by multiple
% threads and ignoring the rest.

% \end{itemize}
