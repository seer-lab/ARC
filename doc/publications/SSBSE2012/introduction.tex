\section{Introduction}
\label{sec:introduction}

As computers and even mobile devices now ship with more than one processor, programs
must take advantage of parallelism or concurrency~\cite{SL05}. Parallelism introduces a new class of bugs
that are difficult to find because they may only occur in a small set of
thread interleavings~\cite{MQB07}. Even when a concurrency bug has been
detected, the repair of such a bug may not be apparent to the developer. This is because many concurrency bugs are the result of the interaction of different code fragments executing in different threads within a program. For example, in a data race there are often multiple threads writing to shared data at the same time.

To address the challenges of detecting and fixing concurrency programs we propose ARC (\textbf{A}utomatic \textbf{R}epair of \textbf{C}oncurrency
bugs) -- an automatic technique to repair deadlocks and data races in concurrent
Java programs. ARC requires no formal specifications or annotations and only the Java source code and tests capable of demonstrating the existing deadlock
and/or data race bugs. ARC works by using evolutionary strategies (ES) to
evolve variants of a ``buggy'' concurrent program until a program is found that fixes the bugs in
question.

The use of search-based software engineering (SBSE)~\cite{Har+10} techniques to
automatically repair bugs is not a novel idea~\cite{FNWG09, WNLF09, NWLF09,
WFGN10, GNFW11, LDFW12}. Our proposed approach adapts the original idea of
automatically fixing \textit{sequential} programs to specifically target
\textit{concurrent} programs. Several challenges exist in concurrent program repair  that are not relevant in sequential source code: (1) concurrent programs can have many possible thread interleavings which can make bugs harder to detect and (2) concurrent programs have to contend with complex interaction bugs that not possible in single-threaded source code. In a recent paper, Goues et al., who developed one of the first approaches to search-based code repair, acknowledged the challenge of fixing concurrent programs -- \textit{``Some properties are difficult or impossible to encode using test cases, such as nondeterministic properties; GenProg cannot currently repair race conditions, for example.''}~\cite{GNFW11}.

As was previously mentioned, ARC uses ES to iteratively evolve a buggy concurrent program into a version that contains the proper synchronization and concurrency mechanisms to resolve any bugs. To evaluate
whether a program is fixed or not, ARC uses IBM's testing tool,
ConTest~\cite{EFN+02}, which allows for the exploration of different interleavings of
the program under test.

A common criticism in automatic bug repair techniques is the large search space, and applying these techniques to  concurrent programs also introduces thread interleavings as well.
ARC innately lessens some of these problems due to a few design choices. First,
we limit the algorithm to only fixing deadlocks and data races. Second, ARC
only targets concurrency mechanisms (\texttt{synchronize} statements are added,
removed, and manipulated). Third, we use a specific set of 11 TXL~\cite{CHP91}
operators based on the ConMAn suite~\cite{BCD06} to mutate the concurrent
program.  Finally, ARC only considers shared variables during the mutation
process. By considering these four design choices ARC reduces the search space and operates only on a limited set
of operators, and a limited set of places where mutation can occur.

In the next section (Sect.~\ref{sec:background}) we cover the background material related to concurrency and evolutionary strategies. The motivation for ARC along with an example
problem are presented in Sect.~\ref{sec:motivation}. In
Sect.~\ref{sec:approach} we describe the approach ARC uses to evolve a fix for
a concurrency bug using ES. We evaluate the fixing effectiveness of ARC in
Sect.~\ref{sec:experiments} by conducing an experiment using several programs.
Three challenges that ARC faces are identified in Sect.~\ref{sec:challenges}.
Threats to validity are discussed in Sect.~\ref{sec:threats}. We describe the
current ongoing research to add a second performance improvement phase to ARC in Sect.~\ref{sec:ongoing}. In
Sect.~\ref{sec:related_works} we discuss related works in the field of
automatic bug repair and finally we present future work and conclusions in
Sect.~\ref{sec:future_work}~\&~\ref{sec:conclusion}.

% In ongoing work we are adding a second phase to the system.  It attempts to
% improve performance by shrinking and removing synchronization blocks. As this
% can introduce data races or deadlocks, any mutant decreasing correctness is
% rejected. This second phase is still in development.  In the rest of the paper
% we concentrate on the first phase, bug fixing.

% To the best of our knowledge there has been no previous work using evolutionary
% strategies to fix bugs in concurrent software. There has been work involving
% the correction of concurrency bugs using self-healing~\cite{LVK08}. From the
% paper, \textit{The healing techniques based on influencing the scheduling do
% not guarantee that a detected problem will really be completely removed, but
% they can decrease the probability of its manifestation.} In contrast ARC is an
% off-line technique that fixes bugs by modifying the source code.

% The main contributions of this paper are:

% \begin{itemize}

% \item An algorithm to create fixes for deadlocks and data races in concurrent Java
% programs. Only the source code and tests demonstrating the bugs are necessary.
% To the best of our knowledge this is the first approach to fix both kinds of
% bugs in Java programs.

% \item Methods to constrain the search space: First, by specifically targeting
% synchronization mechanisms. Second, by using a limited number of TXL operators
% to transform the Java source.  Third, by targeting the variables used by multiple
% threads and ignoring the rest.

% \end{itemize}
