\section{Next Steps in ARC Approach}
\label{sec:ongoing}

ARC may introduce unnecessary concurrency during the fixing process. We are
experimenting with and implementing a second phase (non-functional phase) where
the same kind of evolutionary strategy attempts to optimize the program by
shrinking and removing concurrency blocks (solves issues as illustrated in
Fig.~\ref{fig:optimized_sample_datarace} back in Sect.~\ref{sec:motivation}). A
new fitness function would be required for this non-functional phase:

\begin{footnotesize}
\begin{center}
$non-functional\ fitness(P) = \frac{worst\ score}{[sig_t \times unc(t)] + [sig_c \times unc(c)]}$
\end{center}
\vspace{0.1cm} \textit{Where:} \vspace{0.1cm}
\end{footnotesize}
\begin{scriptsize}
\begin{center}
$unc(x) = \frac{(x_{max} - x_{min})}{x_{avg}}$ \\ \vspace{0.2cm}
$
 sig_t = \left\{
 \begin{array}{l l}
   t/c & \quad if\ t\ > c \\
   c/t & \quad if\ c\ > t \\
 \end{array} \right.
$ \\ \vspace{0.2cm}
$
 sig_s = \left\{
 \begin{array}{l l}
   c/t & \quad if\ t\ > c \\
   t/c & \quad if\ c\ > t \\
 \end{array} \right.
$ \\
\end{center}
\end{scriptsize}


\noindent The proposed fitness function combines the real time ($t$ in fitness function) required for an execution and the number of
voluntary context switches ($c$ in fitness function) made. The real time taken is an obvious choice to
use to measure the performance of the program, while the number of voluntary
context switches gives us an indication on the amount of thread switching. By
minimizing unnecessary synchronization the number of context switches should
decrease along with the real time taken. We plan to acquire a baseline of the
time and context switches from the newly fixed program by running it a large
number of times and averaging the values acquire. This baseline is then used in
the fitness function to evaluate relative improvements from the fixed
unoptimized program. Note that the fitness function would adjust based on the
significance and uncertainty of both variables (to be fair in situations where
either variable is under-represented).

Removing and reducing synchronization runs the risk of introducing new bugs
into the program. Before every non-functional phase's evaluation we need to
ensure that no bugs are present, thus we re-run the first phase's evaluation
with the multiplier on the number of runs. If any deadlocks or data races are
encountered the proposed optimization is rejected and this individual is reset
to the previous generation. After ARC validates the proposed optimization
additional runs are conducted without using ConTest (avoiding the inclusion of
random noise) to obtain the values required for the non-functional fitness
function. Currently, we have implemented most of this proposed extension but have not conducted any empirical evaluation.
