\section{Related Works}
\label{sec:related_works}

There are a number of approaches to automatically hide or fix bugs in programs.
Co-evolutionary competition between programs with bugs and test cases is used
in~\cite{AY08, Arc08, WT10}. Both approaches require formal specifications and
both use genetic programming to evolve fixes.
%Co-evolution is used to cause competition between the evolving fixes and test cases.
%In both cases the
%resulting framework is tested only against a toy sorting algorithm with mixed
%success. The untargeted nature of the fixing process is the largest limitation.
%No effort is made to constrain the search space of the problem.

Significant advances are made in~\cite{FNWG09, WNLF09, NWLF09, WFGN10, GNFW11,
LDFW12}. Formal specifications are not required. Instead, test cases are used
to demonstrate the bug and describe the desired functionality that must be
preserved. To address the limitations of the previous approach they introduce
two innovations: First, they assume the bug is written correctly in another
part of the program. Second, they determine the error path on which the bug
occurs and target those statements specifically for repair. Together these
additions constrain the state space enough that the framework can fix real bugs
in real programs.

Similar to the work done here,~\cite{KLT+07, LVK08} uses ConTest to heal data
races. From their paper, \textit{``Healing concurrency problems is about
limiting or changing the probability of interleaving, such that bugs will be
seen less.''}
%In contrast, our approach uses ConTest in a framework that fixes
%both deadlocks and data races.

A framework is created in~\cite{CB05} to ``repair'' buffer overflow attacks.
When one is identified, the program state is reset to what it was before the
attack. The attack packet is discarded then the program continues running.
%As in the previous approach the goal is to hide the problem, not fix it.

SAT solving is used in~\cite{AY07}  to repair shared memory concurrent programs
``w.r.t. CTL specifications'' where processes atomically read, write one shared
variable at a time. All that is required is the concurrent program and a
specification in modal or temporal logic.
%It is unclear if the fix is applied to the program or the specification of the program.

A number of tools are brought together in~\cite{JSZL+11} to fix single variable atomicity violations. From dynamic bug analysis to patch creation and merging and finally dynamic testing it covers the entire bug fixing lifecycle.