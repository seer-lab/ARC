\section{Related Works}
\label{sec:related_works}

%There are a number of approaches to automatically hide or fix bugs in programs.

With respect to sequential programs, several approaches to program repair have been proposed in the literature. For example, co-evolutionary competition between programs with bugs and between test cases is used
in~\cite{AY08, Arc08, WT10}. Both approaches require formal specifications and
both use genetic programming to evolve fixes. Alternatively the work in~\cite{FNWG09, WNLF09, NWLF09, WFGN10, GNFW11,
LDFW12} requires no formal specifications. Instead, test cases are used
to demonstrate the bug and describe the desired functionality that must be
preserved. To address the limitations of the previous approach they introduce
two innovations: First, they assume the bug is written correctly in another
part of the program. Second, they determine the error path on which the bug
occurs and target those statements specifically for repair. Together these
additions constrain the state space enough that the framework can fix real bugs
in real programs.

With respect to concurrent programs there are several examples of related work. Similar to the work done here,~\cite{KLT+07, LVK08} uses ConTest to heal data
races. Healing a program is not the same as repairing a program -- \textit{``Healing concurrency problems is about
limiting or changing the probability of interleaving, such that bugs will be
seen less.''} SAT solving is used in~\cite{AY07}  to repair shared memory concurrent programs
``w.r.t. CTL specifications'' where processes atomically read, write one shared
variable at a time. This work requires the concurrent program and a formal temporal logic
specification. Single variable atomicity violations are repaired in~\cite{JSZL+11}. This approach combines dynamic bug analysis, patch creation and merging and dynamic testing but is more limited then our approach with respect to the kinds of bugs that can be repaired.