\section{Motivation}
\label{sec:motivation}

Recall that concurrency bugs often are the result of multiple code fragments in different threads interacting in an undesirable way. The multiple code fragments involved in a concurrency bug can be
in different units of code and can therefore be difficult to locate, making the appropriate fix not
always clear. Once solution to this problem is to use concurrency anti-patterns that provide a
definition, problem, context and solution for concurrency
bugs~\cite{BJ09,FKLV12}. Unfortunately, the use of anti-patterns still requires the programmer to have identified the concurrency bug before assisting with repairing the bug.

\begin{figure}[t!]
\begin{minipage}{5cm}
\footnotesize{\textbf{Buggy Program:}}
\begin{lstlisting}[language=Java, morekeywords={synchronize}]
write(int var1){
  ... // Expensive loop
  data = var1;
  ... // Database query
}

int public read(){
  return data;
}
\end{lstlisting}
\end{minipage}\hfill
\begin{minipage}{5cm}
\footnotesize{\textbf{Fixed Program:}}
\begin{lstlisting}[language=Java, morekeywords={synchronize}]
synchronize write(int var1){
  ... // Expensive loop
  data = var1;
  ... // Database query
}

int synchronize read(){
  return data;
}
\end{lstlisting}
\end{minipage}
\caption{A developer first synchronizes the \texttt{read} function, yet the bug
still exists. Synchronizing the \texttt{write} method as well fixes the bug.}
\label{fig:fixed_sample_datarace}
\end{figure}

\begin{figure}[t!]
\begin{minipage}{5cm}
\footnotesize{\textbf{$1^{st}$ Optimization on Fix:}}
\begin{lstlisting}[language=Java, morekeywords={synchronize}]
public write(int var1){
 ... // Expensive loop
 synchronized(this){
   data = var1;
   ... // Database query
 }
}

int synchronize read(){
 return data;
}
\end{lstlisting}
\end{minipage}\hfill
\begin{minipage}{5cm}
\footnotesize{\textbf{$2^{nd}$ Optimization on Fix:}}
\begin{lstlisting}[language=Java, morekeywords={synchronize}]
public write(int var1){
 ... // Expensive loop
 synchronized(this){
   data = var1;
 }
 ... // Database query
}

int synchronize read(){
 return data;
}
\end{lstlisting}
\end{minipage}
\caption{A developer shrinks the critical region to exclude the expensive loop (Optimization 1).
Next, a developer shrink the critical region again to exclude the database query (Optimization 2).}
\label{fig:optimized_sample_datarace}
\end{figure}


To illustrate the challenges of concurrency bug repair we now consider an example of a data race bug and how a developer might fix it. In the left part of Fig.~\ref{fig:fixed_sample_datarace} the
\texttt{read} and \texttt{write} method access a shared variable. A very simple
data race exists because there is no atomic access to the \texttt{data}
variable during the concurrent reading or writing.
%Both methods are involved in the data race, and it is because of the interactions between these methods the data race is possible.
A possible repair involves synchronizing both accesses as shown in the right part of
Fig.~\ref{fig:fixed_sample_datarace}. Note that synchronizing one method alone does not
completely fix the bug in this case.

% Phase 2 - the optimization phase in development - addresses the issue of unnecessarily
% synchronizing the ``expensive loop'' and ``database query''.

The solution in the right part of Fig.~\ref{fig:fixed_sample_datarace} is far
from ideal. It forces other threads to wait unnecessarily while the write
method works in the loop and database sections. An optimization is to shrink
the critical region (the synchronized statements) to only guard access to the
shared variable as shown in Fig.~\ref{fig:optimized_sample_datarace}.
%From a developer's standpoint there is a lot of work involved in creating a fix for parallel bugs as multiple, unrelated changes are not uncommon. 
In total two changes
are required to functionally fix the example program, and two additional
changes are required to improve the non-functional performance.% An ideal fix only requires two changes which minimizes the time spent in the critical region. 
Currently, ARC is capable of finding a fix for a concurrency bug, it does not make an
attempt at ensuring the fix is close to optimal (finding the best fix from a performance standpoint is ongoing research and is explained in Sect.~\ref{sec:ongoing}).

%Automated tools in software testing and debugging are needed as they have the potential to reduce the vast amount of resources spent on software testing (upwards to \$59.5 billion)~\cite{RTI02}.
%ARC provides an automated approach to fixing the \textit{functionality}
%, and then optimizing the \textit{non-functional} performance
%of programs with deadlocks and data races.
