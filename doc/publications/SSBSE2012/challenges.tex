% \section{Challenges}
% \label{sec:challenges}

% ARC has three major challenges to face:

% \textbf{Large Solution Space}: Controlling the search space is critical to ARC's
% success.  We are only considering
% deadlocks and data races and are only mutating concurrency mechanisms with concurrency
% mutation operators. This significantly shrinks the search space to explore
% to find a fix. Operator weighting and only using variables involved in concurrency
% further constrain the space.
% %In Arcuri's approach~\cite{AY08}, they were
% %considering the whole program along with a large set of mutations. When
% %combined with a large number of places to apply them it creates an extremely
% %large search space for finding the solution.

% The search space can be reduced even further by considering only the mutations
% falling on the error path (Used in Weimer's approach~\cite{GNFW11}) or even by
% using external detection tools for potential a locational where the bug might
% be residing.
% %These two suggestions are considered for future work to minimize
% %the solution space.

% \textbf{Dependence on Test Suite}: ARC depends on a minimal test suite for the
% program being repaired. In both Arcuri's and Weimer's
% approach~\cite{AY08,GNFW11} the test suites are used to both guard the
% program's functionality and to indicate a bug is present. ARC requires a test
% suit to exhibit the bug. ARC currently has no solution for dealing with bugs
% that occur outside of the test suite's coverage.
% %Arcuri's and Weimer's
% %approaches also has no solution for this.

% %There has been some work that attempts to co-evolve the test suite at the same
% %time to help check for new situations~\cite{WT10}. We assume the
% %test suite is comprehensive enough to detect all bugs. ARC will eventually only
% %consider using mutation operators that fall within the coverage of the test
% %suite to avoid introducing new bugs the test suite cannot detect.

% \textbf{Program Readability}: Automatically fixing a program
% raises the question of source readability. This is a concern as developers want to
% understand the fix as well. All TXL operators were constructed to follow Java's
% coding conventions and pretty-print all code on output.

% %Posnett et. al.~\cite{PHD11} use the following
% %metrics to estimate readability:

% %\begin{itemize}
%  % \item Moving lines around does not affect the program's readability
%  % \item The addition of existing tokens does not reduce the program's
% %readability as much as adding new unique statements
% %\end{itemize}

% %ARC simply adds, removes manipulates synchronization statements.
% %In most cases
% %there will already tokens regarding synchronization within the program. Thus,
% %the addition of new synchronization does not reduce the program's readability
% %that much. For the mutations that manipulate the synchronization statements ARC
% %is effectively moving lines around, which does not reduce the program's
% %readability.

% % Readability metrics for programs fixed by ARC are collected in
% % Table~\ref{tbl:readability}.
% % \ldots % TODO

% % \begin{table}[h]
% % \caption{The program readability values from the buggy program to the fixed
% % program to the optimized program.}
% % \begin{center}
% % \begin{tabular}{|l|p{2cm}|p{2cm}|p{2cm}|p{2cm}|p{2cm}|}
% % \hline
% % \textbf{Program} & \textbf{Buggy Program's Readability} & \textbf{Fixed Program's Readability} & \textbf{Optimized Program's Readability} & \textbf{Change from Buggy to Fixed} & \textbf{Change from Fixed to Optimized}\\
% % \hline
% % account & 100 & 110 & 105 & +10\% & -4.6\%\\
% % \hline
% % accounts & 100 & 110 & 105 & +10\% & -4.6\%\\
% % \hline
% % airline & 100 & 110 & 105 & +10\% & -4.6\%\\
% % \hline
% % allocation & 100 & 110 & 105 & +10\% & -4.6\%\\
% % \hline
% % bubble & 100 & 110 & 105 & +10\% & -4.6\%\\
% % \hline
% % bubblesort2 & 100 & 110 & 105 & +10\% & -4.6\%\\
% % \hline
% % buffer & 100 & 110 & 105 & +10\% & -4.6\%\\
% % \hline
% % bufwriter & 100 & 110 & 105 & +10\% & -4.6\%\\
% % \hline
% % deadlock & 100 & 110 & 105 & +10\% & -4.6\%\\
% % \hline
% % lottery & 100 & 110 & 105 & +10\% & -4.6\%\\
% % \hline
% % mergesort & 100 & 110 & 105 & +10\% & -4.6\%\\
% % \hline
% % pingpong & 100 & 110 & 105 & +10\% & -4.6\%\\
% % \hline
% % \end{tabular}
% % \label{tbl:readability}
% % \end{center}
% % \end{table}
