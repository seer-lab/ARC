
%% bare_conf.tex
%% V1.3
%% 2007/01/11
%% by Michael Shell
%% See:
%% http://www.michaelshell.org/
%% for current contact information.
%%
%% This is a skeleton file demonstrating the use of IEEEtran.cls
%% (requires IEEEtran.cls version 1.7 or later) with an IEEE conference paper.
%%
%% Support sites:
%% http://www.michaelshell.org/tex/ieeetran/
%% http://www.ctan.org/tex-archive/macros/latex/contrib/IEEEtran/
%% and
%% http://www.ieee.org/

%%*************************************************************************
%% Legal Notice:
%% This code is offered as-is without any warranty either expressed or
%% implied; without even the implied warranty of MERCHANTABILITY or
%% FITNESS FOR A PARTICULAR PURPOSE!
%% User assumes all risk.
%% In no event shall IEEE or any contributor to this code be liable for
%% any damages or losses, including, but not limited to, incidental,
%% consequential, or any other damages, resulting from the use or misuse
%% of any information contained here.
%%
%% All comments are the opinions of their respective authors and are not
%% necessarily endorsed by the IEEE.
%%
%% This work is distributed under the LaTeX Project Public License (LPPL)
%% ( http://www.latex-project.org/ ) version 1.3, and may be freely used,
%% distributed and modified. A copy of the LPPL, version 1.3, is included
%% in the base LaTeX documentation of all distributions of LaTeX released
%% 2003/12/01 or later.
%% Retain all contribution notices and credits.
%% ** Modified files should be clearly indicated as such, including  **
%% ** renaming them and changing author support contact information. **
%%
%% File list of work: IEEEtran.cls, IEEEtran_HOWTO.pdf, bare_adv.tex,
%%                    bare_conf.tex, bare_jrnl.tex, bare_jrnl_compsoc.tex
%%*************************************************************************

% *** Authors should verify (and, if needed, correct) their LaTeX system  ***
% *** with the testflow diagnostic prior to trusting their LaTeX platform ***
% *** with production work. IEEE's font choices can trigger bugs that do  ***
% *** not appear when using other class files.                            ***
% The testflow support page is at:
% http://www.michaelshell.org/tex/testflow/



% Note that the a4paper option is mainly intended so that authors in
% countries using A4 can easily print to A4 and see how their papers will
% look in print - the typesetting of the document will not typically be
% affected with changes in paper size (but the bottom and side margins will).
% Use the testflow package mentioned above to verify correct handling of
% both paper sizes by the user's LaTeX system.
%
% Also note that the "draftcls" or "draftclsnofoot", not "draft", option
% should be used if it is desired that the figures are to be displayed in
% draft mode.
%
\documentclass[10pt, conference, compsocconf]{IEEEtran}
% Add the compsocconf option for Computer Society conferences.
%
% If IEEEtran.cls has not been installed into the LaTeX system files,
% manually specify the path to it like:
% \documentclass[conference]{../sty/IEEEtran}





% Some very useful LaTeX packages include:
% (uncomment the ones you want to load)
\usepackage{color}
\usepackage{colortbl}

\definecolor{lightgray}{gray}{.95}
\definecolor{darkgray}{gray}{.80}

\usepackage{epsfig}
\usepackage{listings}

\usepackage{array}
\usepackage{longtable}
\usepackage{hhline}
\usepackage{listings}
%\usepackage{tabular}

\lstset{ %
lineskip=0.5pt,
basicstyle=\scriptsize,       % the size of the fonts that are used for the code
%numbers=left,                   % where to put the line-numbers
numberstyle=\scriptsize,      % the size of the fonts that are used for the line-numbers
stepnumber=1,                   % the step between two line-numbers. If it is 1 each line will be numbered
numbersep=3pt,                  % how far the line-numbers are from the code
backgroundcolor=\color{lightgray},  % choose the background color. You must add \usepackage{color}
showspaces=false,               % show spaces adding particular underscores
showstringspaces=false,         % underline spaces within strings
showtabs=false,                 % show tabs within strings adding particular underscores
frame=none,                   % adds a frame around the code
tabsize=2,              % sets default tabsize to 2 spaces
captionpos=b,                   % sets the caption-position to bottom
breaklines=true,        % sets automatic line breaking
breakatwhitespace=false,    % sets if automatic breaks should only happen at whitespace
escapeinside={\%}{)}          % if you want to add a comment within your code
}


% *** MISC UTILITY PACKAGES ***
%
%\usepackage{ifpdf}
% Heiko Oberdiek's ifpdf.sty is very useful if you need conditional
% compilation based on whether the output is pdf or dvi.
% usage:
% \ifpdf
%   % pdf code
% \else
%   % dvi code
% \fi
% The latest version of ifpdf.sty can be obtained from:
% http://www.ctan.org/tex-archive/macros/latex/contrib/oberdiek/
% Also, note that IEEEtran.cls V1.7 and later provides a builtin
% \ifCLASSINFOpdf conditional that works the same way.
% When switching from latex to pdflatex and vice-versa, the compiler may
% have to be run twice to clear warning/error messages.






% *** CITATION PACKAGES ***
%
\usepackage{cite}
% cite.sty was written by Donald Arseneau
% V1.6 and later of IEEEtran pre-defines the format of the cite.sty package
% \cite{} output to follow that of IEEE. Loading the cite package will
% result in citation numbers being automatically sorted and properly
% "compressed/ranged". e.g., [1], [9], [2], [7], [5], [6] without using
% cite.sty will become [1], [2], [5]--[7], [9] using cite.sty. cite.sty's
% \cite will automatically add leading space, if needed. Use cite.sty's
% noadjust option (cite.sty V3.8 and later) if you want to turn this off.
% cite.sty is already installed on most LaTeX systems. Be sure and use
% version 4.0 (2003-05-27) and later if using hyperref.sty. cite.sty does
% not currently provide for hyperlinked citations.
% The latest version can be obtained at:
% http://www.ctan.org/tex-archive/macros/latex/contrib/cite/
% The documentation is contained in the cite.sty file itself.






% *** GRAPHICS RELATED PACKAGES ***
%
\ifCLASSINFOpdf
  % \usepackage[pdftex]{graphicx}
  % declare the path(s) where your graphic files are
  % \graphicspath{{../pdf/}{../jpeg/}}
  % and their extensions so you won't have to specify these with
  % every instance of \includegraphics
  % \DeclareGraphicsExtensions{.pdf,.jpeg,.png}
\else
  % or other class option (dvipsone, dvipdf, if not using dvips). graphicx
  % will default to the driver specified in the system graphics.cfg if no
  % driver is specified.
  % \usepackage[dvips]{graphicx}
  % declare the path(s) where your graphic files are
  % \graphicspath{{../eps/}}
  % and their extensions so you won't have to specify these with
  % every instance of \includegraphics
  % \DeclareGraphicsExtensions{.eps}
\fi
% graphicx was written by David Carlisle and Sebastian Rahtz. It is
% required if you want graphics, photos, etc. graphicx.sty is already
% installed on most LaTeX systems. The latest version and documentation can
% be obtained at:
% http://www.ctan.org/tex-archive/macros/latex/required/graphics/
% Another good source of documentation is "Using Imported Graphics in
% LaTeX2e" by Keith Reckdahl which can be found as epslatex.ps or
% epslatex.pdf at: http://www.ctan.org/tex-archive/info/
%
% latex, and pdflatex in dvi mode, support graphics in encapsulated
% postscript (.eps) format. pdflatex in pdf mode supports graphics
% in .pdf, .jpeg, .png and .mps (metapost) formats. Users should ensure
% that all non-photo figures use a vector format (.eps, .pdf, .mps) and
% not a bitmapped formats (.jpeg, .png). IEEE frowns on bitmapped formats
% which can result in "jaggedy"/blurry rendering of lines and letters as
% well as large increases in file sizes.
%
% You can find documentation about the pdfTeX application at:
% http://www.tug.org/applications/pdftex





% *** MATH PACKAGES ***
%
%\usepackage[cmex10]{amsmath}
% A popular package from the American Mathematical Society that provides
% many useful and powerful commands for dealing with mathematics. If using
% it, be sure to load this package with the cmex10 option to ensure that
% only type 1 fonts will utilized at all point sizes. Without this option,
% it is possible that some math symbols, particularly those within
% footnotes, will be rendered in bitmap form which will result in a
% document that can not be IEEE Xplore compliant!
%
% Also, note that the amsmath package sets \interdisplaylinepenalty to 10000
% thus preventing page breaks from occurring within multiline equations. Use:
%\interdisplaylinepenalty=2500
% after loading amsmath to restore such page breaks as IEEEtran.cls normally
% does. amsmath.sty is already installed on most LaTeX systems. The latest
% version and documentation can be obtained at:
% http://www.ctan.org/tex-archive/macros/latex/required/amslatex/math/





% *** SPECIALIZED LIST PACKAGES ***
%
%\usepackage{algorithmic}
% algorithmic.sty was written by Peter Williams and Rogerio Brito.
% This package provides an algorithmic environment fo describing algorithms.
% You can use the algorithmic environment in-text or within a figure
% environment to provide for a floating algorithm. Do NOT use the algorithm
% floating environment provided by algorithm.sty (by the same authors) or
% algorithm2e.sty (by Christophe Fiorio) as IEEE does not use dedicated
% algorithm float types and packages that provide these will not provide
% correct IEEE style captions. The latest version and documentation of
% algorithmic.sty can be obtained at:
% http://www.ctan.org/tex-archive/macros/latex/contrib/algorithms/
% There is also a support site at:
% http://algorithms.berlios.de/index.html
% Also of interest may be the (relatively newer and more customizable)
% algorithmicx.sty package by Szasz Janos:
% http://www.ctan.org/tex-archive/macros/latex/contrib/algorithmicx/




% *** ALIGNMENT PACKAGES ***
%
%\usepackage{array}
% Frank Mittelbach's and David Carlisle's array.sty patches and improves
% the standard LaTeX2e array and tabular environments to provide better
% appearance and additional user controls. As the default LaTeX2e table
% generation code is lacking to the point of almost being broken with
% respect to the quality of the end results, all users are strongly
% advised to use an enhanced (at the very least that provided by array.sty)
% set of table tools. array.sty is already installed on most systems. The
% latest version and documentation can be obtained at:
% http://www.ctan.org/tex-archive/macros/latex/required/tools/


%\usepackage{mdwmath}
%\usepackage{mdwtab}
% Also highly recommended is Mark Wooding's extremely powerful MDW tools,
% especially mdwmath.sty and mdwtab.sty which are used to format equations
% and tables, respectively. The MDWtools set is already installed on most
% LaTeX systems. The lastest version and documentation is available at:
% http://www.ctan.org/tex-archive/macros/latex/contrib/mdwtools/


% IEEEtran contains the IEEEeqnarray family of commands that can be used to
% generate multiline equations as well as matrices, tables, etc., of high
% quality.


%\usepackage{eqparbox}
% Also of notable interest is Scott Pakin's eqparbox package for creating
% (automatically sized) equal width boxes - aka "natural width parboxes".
% Available at:
% http://www.ctan.org/tex-archive/macros/latex/contrib/eqparbox/





% *** SUBFIGURE PACKAGES ***
%\usepackage[tight,footnotesize]{subfigure}
% subfigure.sty was written by Steven Douglas Cochran. This package makes it
% easy to put subfigures in your figures. e.g., "Figure 1a and 1b". For IEEE
% work, it is a good idea to load it with the tight package option to reduce
% the amount of white space around the subfigures. subfigure.sty is already
% installed on most LaTeX systems. The latest version and documentation can
% be obtained at:
% http://www.ctan.org/tex-archive/obsolete/macros/latex/contrib/subfigure/
% subfigure.sty has been superceeded by subfig.sty.



%\usepackage[caption=false]{caption}
%\usepackage[font=footnotesize]{subfig}
% subfig.sty, also written by Steven Douglas Cochran, is the modern
% replacement for subfigure.sty. However, subfig.sty requires and
% automatically loads Axel Sommerfeldt's caption.sty which will override
% IEEEtran.cls handling of captions and this will result in nonIEEE style
% figure/table captions. To prevent this problem, be sure and preload
% caption.sty with its "caption=false" package option. This is will preserve
% IEEEtran.cls handing of captions. Version 1.3 (2005/06/28) and later
% (recommended due to many improvements over 1.2) of subfig.sty supports
% the caption=false option directly:
%\usepackage[caption=false,font=footnotesize]{subfig}
%
% The latest version and documentation can be obtained at:
% http://www.ctan.org/tex-archive/macros/latex/contrib/subfig/
% The latest version and documentation of caption.sty can be obtained at:
% http://www.ctan.org/tex-archive/macros/latex/contrib/caption/




% *** FLOAT PACKAGES ***
%
%\usepackage{fixltx2e}
% fixltx2e, the successor to the earlier fix2col.sty, was written by
% Frank Mittelbach and David Carlisle. This package corrects a few problems
% in the LaTeX2e kernel, the most notable of which is that in current
% LaTeX2e releases, the ordering of single and double column floats is not
% guaranteed to be preserved. Thus, an unpatched LaTeX2e can allow a
% single column figure to be placed prior to an earlier double column
% figure. The latest version and documentation can be found at:
% http://www.ctan.org/tex-archive/macros/latex/base/



%\usepackage{stfloats}
% stfloats.sty was written by Sigitas Tolusis. This package gives LaTeX2e
% the ability to do double column floats at the bottom of the page as well
% as the top. (e.g., "\begin{figure*}[!b]" is not normally possible in
% LaTeX2e). It also provides a command:
%\fnbelowfloat
% to enable the placement of footnotes below bottom floats (the standard
% LaTeX2e kernel puts them above bottom floats). This is an invasive package
% which rewrites many portions of the LaTeX2e float routines. It may not work
% with other packages that modify the LaTeX2e float routines. The latest
% version and documentation can be obtained at:
% http://www.ctan.org/tex-archive/macros/latex/contrib/sttools/
% Documentation is contained in the stfloats.sty comments as well as in the
% presfull.pdf file. Do not use the stfloats baselinefloat ability as IEEE
% does not allow \baselineskip to stretch. Authors submitting work to the
% IEEE should note that IEEE rarely uses double column equations and
% that authors should try to avoid such use. Do not be tempted to use the
% cuted.sty or midfloat.sty packages (also by Sigitas Tolusis) as IEEE does
% not format its papers in such ways.



\usepackage{boxedminipage}
\usepackage{graphicx}

\newcommand{\notesbox}[1]{
%     \ \\
      \noindent\begin{center}\begin{boxedminipage}[h]{0.4\textwidth}{#1}\end{boxedminipage}\end{center}
}

% *** PDF, URL AND HYPERLINK PACKAGES ***
%
%\usepackage{url}
% url.sty was written by Donald Arseneau. It provides better support for
% handling and breaking URLs. url.sty is already installed on most LaTeX
% systems. The latest version can be obtained at:
% http://www.ctan.org/tex-archive/macros/latex/contrib/misc/
% Read the url.sty source comments for usage information. Basically,
% \url{my_url_here}.

% *** Do not adjust lengths that control margins, column widths, etc. ***
% *** Do not use packages that alter fonts (such as pslatex).         ***
% There should be no need to do such things with IEEEtran.cls V1.6 and later.
% (Unless specifically asked to do so by the journal or conference you plan
% to submit to, of course. )


% correct bad hyphenation here
\hyphenation{op-tical net-works semi-conduc-tor}


\begin{document}
%
% paper title
% can use linebreaks \\ within to get better formatting as desired
\title{ARC - Automatic Repair of Concurrency Bugs}


% author names and affiliations
% use a multiple column layout for up to two different
% affiliations

\author{\IEEEauthorblockN{Kevin Jalbert, David Kelk, Jeremy S. Bradbury}
\IEEEauthorblockA{Software Quality Research Group\\
Faculty of Science (Computer Science)\\
University of Ontario Institute of Technology\\
Oshawa, Ontario, Canada\\
\{kevin.jalbert, david.kelk, jeremy.bradbury\}@uoit.ca}
}

% conference papers do not typically use \thanks and this command
% is locked out in conference mode. If really needed, such as for
% the acknowledgment of grants, issue a \IEEEoverridecommandlockouts
% after \documentclass

% for over three affiliations, or if they all won't fit within the width
% of the page, use this alternative format:
%
%\author{\IEEEauthorblockN{Michael Shell\IEEEauthorrefmark{1},
%Homer Simpson\IEEEauthorrefmark{2},
%James Kirk\IEEEauthorrefmark{3},
%Montgomery Scott\IEEEauthorrefmark{3} and
%Eldon Tyrell\IEEEauthorrefmark{4}}
%\IEEEauthorblockA{\IEEEauthorrefmark{1}School of Electrical and Computer Engineering\\
%Georgia Institute of Technology,
%Atlanta, Georgia 30332--0250\\ Email: see http://www.michaelshell.org/contact.html}
%\IEEEauthorblockA{\IEEEauthorrefmark{2}Twentieth Century Fox, Springfield, USA\\
%Email: homer@thesimpsons.com}
%\IEEEauthorblockA{\IEEEauthorrefmark{3}Starfleet Academy, San Francisco, California 96678-2391\\
%Telephone: (800) 555--1212, Fax: (888) 555--1212}
%\IEEEauthorblockA{\IEEEauthorrefmark{4}Tyrell Inc., 123 Replicant Street, Los Angeles, California 90210--4321}}




% use for special paper notices
%\IEEEspecialpapernotice{(Invited Paper)}




% make the title area
\maketitle


\begin{abstract}

Automatic repair of single-threaded programs is starting to be realised in practice.  Similar progress has not been made for the automatic repair of parallel programs.  We introduce a fully automated two-phase system for repairing deadlocks and data races in parallel Java programs. The approach works on any Java source code and requires only rudimentary test cases. Annotations, formal specifications or other notations are not required. As only the concurrency mechanisms are targeted the semantic meaning of the program is not changed. In the first phase an evolutionary strategy is used to mutate the existing program until a variant is found that  fixes the deadlock or data race. As the first phase may introduce unneeded synchronization, the second phase attempts to optimize performance by removing the excess synchronization without sacraficing program correctness. We describe the approach and report on early results.  

\end{abstract}

\begin{IEEEkeywords}
concurrency; genetic programming; mutation.

\end{IEEEkeywords}


% For peer review papers, you can put extra information on the cover
% page as needed:
% \ifCLASSOPTIONpeerreview
% \begin{center} \bfseries EDICS Category: 3-BBND \end{center}
% \fi
%
% For peerreview papers, this IEEEtran command inserts a page break and
% creates the second title. It will be ignored for other modes.
\IEEEpeerreviewmaketitle



\section{Introduction}
\label{sec:introduction}

%%%TODO Integrate Me
Research and development surrounding multi-threaded software systems has
recently boomed due to the increased demand and attention for concurrent
software~\cite{SL05}. Concurrency has the benefit of increased performance by
exploiting hardware advances in multi-cores. The largest issue associated with
concurrent software systems is the \textit{heisenbug}, a difficult to detect
bug hidden by the way threads are scheduled~\cite{MQB07}.
%%%%

As desktop computers now ship with more than one processor, programs must
parallelize to continue to benefit from Moore's Law. Inevitibly these programs
will contain bugs. If fixing bugs in single-threaded programs is a difficult
and resource intensive task,  it is doubly so in multi-threaded programs.
Parallelism introduces new classes of bugs and makes them harder to find by
having them only occur in rare execution interleavings for example. Data-races
cause variables to take on unpredictable values as different threads race to
write to them while deadlocks bring part or all of a running program to a halt.

We propose ARC (Automatic Repair of Concurrency bugs): An automatic technique
to repair deadlocks and data races in parallel Java programs. Formal
specifications, annotations and elaborate test suites are not required. Only
the Java source code and test(s) demonstrating the deadlocks and data races are
necessary. Evolutionary strategies are used to evolve variants until one is
found that fixes the bugs in question.

There has been a great deal of research in the area of search-based software
engineering~\cite{Har+10}. Furthermore, the use of heuristic search to identify
a solution that fixes a bug is not a novel idea~\cite{FNWG09, AY08, Arc08,
WT10, WNLF09, WFGN10}. Our proposed approach adapts the original idea of
automatically fixing sequential software to specifically target concurrent
software.

Evolutionary strategies (ES) is part of the family of heuristic search
algorithms. They are population-based technique driven by mutation. Crossover
and replacement are not used.  All members of the population exist throughout
an invocation of the search. A fitness function is used to evaluate each
member's proposed fix. Bugs in a concurrent program are fixed by iteratively
mutating the program and evaluating each mutation by executing the program a
number of times. CONTest (cite) is used to explore different interleavings of
the program under test to increase our confidence bugs are not slipping by.
Fitness improves with the number of correct executions.

Large search spaces are a problem faced by all bug fixing techniques.
Parallelism introduces thread interleavings on top of this. A number of steps
are taken to address this issue.  First, we limit the algorithm to only fixing
deadlocks and data races. Second, only the concurrency mechanisms are targeted.
Specifically, \textit{synchronize} statements are added, removed, swapped,
grown and shrunk. No other statements are affected. Third, we use a specific
set of eleven TXL operators (Mention ConMan here?) operators to mutate the
code.

ARC operates in two phases. In the first, synchronized blocks are added,
expanded and swapped to attempt to fix the bugs. This may add unnecessary
synchronization. If a correct program is found in phase 1 a second phase is run
where synchronization blocks are shrunk or removed in an attempt to increase
efficiency. As this can introduce data races or deadlocks, any mutant
decreasing correctness is rejected.

To the best of our knowledge there has been no previous work using evolutionary
strategies to fix bugs in concurrent software.  There has been work involving
the correction of concurrency bugs using self-healing~\cite{LVK08}. From the
paper, \textit{The healing techniques based on influencing the scheduling do
not guarantee that a detected problem will really be completely removed, but
they can decrease the probability of its manifestation.} In contrast ARC is an
off-line technique that fixes the bug by modifying the source code.

The main contributions of this paper are:

\begin{itemize}

\item An algorithm to create minimal fixes for deadlocks and data races in Java
programs. Only the source code and tests demonstrating the bugs are necessary.
To the best of our knowledge this is the first approach to fix both kinds of
bugs in Java programs.

\item Methods to constrain the search space: Specific targeting of
synchronization mechanisms and the TXL operators implementing this.

\end{itemize}




\subsection{Motivation using quote from GenProg paper}




\section{Background}
\label{sec:background}
All of the material discussed in this paper revolves around topics of
concurrency, model checking and searches to explore state-spaces. A firm
understanding of these topics are required for various parts of the paper, thus
some background information is provided.

\subsection{Concurrency}
\label{sec:concurrency}
Concurrency is the act of having multiple threads executing in parallel.
Threads in programs are objects which execute statements, thus by having
multiple threads in a program it is possible to executing statements in
parallel. Concurrent programs are able to exploit multi-core systems since
threads can be distributed across each of the CPUs to increase the performance.
The inherit problem with concurrency is that a new class of bugs have emerged,
the \textit{heisenbugs}, which are tied to the concurrent nature of
multi-threads. Due to concurrent access to shared memory, threads are able to
cause \textit{dataraces} and \textit{deadlocks}.

A \textbf{data race} has been defined as: \textit{``\ldots two or more
concurrent threads access a shared variable and when at least one access is a
write, and the threads use no explicit mechanism to prevent the access from
being simultaneous.''}~\cite{LSW07}

A \textbf{deadlock} has been defined as: \textit{``\ldots a situation where two
or more processes are unable to proceed because each is waiting for one of the
others to do something in a deadlock cycle \ldots} For example, this occurs
when a thread holds a lock that another thread desires and
vice-versa''~\cite{LSW07}

These bugs are extremely difficult to detect due to the non-deterministic
nature of how the threads are interleaved (the way the system schedules them).
In one execution a concurrency bug might occur, then a second execution the bug
might not occur due to how the threads were scheduled. Various techniques are
available to try and detect concurrency bugs such as static
analysis~\cite{NA07,NPSG09,HP04}, stress testing~\cite{HSU03}, dynamic
analysis~\cite{JNPS09,EFN+02}, and model
checking~\cite{BHPV00,RDH03,OM03,MQB07,Holz97,JM04,BHPV00,HP00}.

\subsection{Evolutionary Algorithms}

\section{Related Works}

TODO Previous attempts of Auto. Fixing bugs

\section{Motivating Problem}

\section{Our solution -- Walkthrough -- Functional Phase}

\begin{figure}[!t]
  \centering
  %\hrule
  %\vspace{2mm}
  \includegraphics[width=6.7cm]{figures/process.pdf}
  \caption{ARC's Repair and Optimization Process}
  \label{fig:process}
  \vspace{1mm}
  %\footnotesize{Using the repaired program $P'$, an optimization phase occurs to minimize unnecessary synchronizations.}
  %\vspace{2mm}
  %\hrule
\end{figure}



\subsection{Program Representation}
\label{sec:programrepresentation}



\subsection{Evolutionary Strategy}
\label{sec:evolutionarystrategy}



\subsubsection{Heuristic Aspects}
\label{sec:heuristicaspects}



\subsection{Mutation Operators}
\label{sec:mutationoperators}

\begin{table}
\begin{center}
\begin{tabular}{|l|l|c|c|}
\hline
\textbf{Operator} &
\textbf{Acronym} &
\textbf{Functional} &
\textbf{Non-Functional}
\\\hline

Add synch. around synch. & ASAS & $\surd$ &
\\\hline

Add synch. around variable & ASAV & $\surd$ &
\\\hline

Add synch. around method & ASM & $\surd$ &
\\\hline

Change synch. order & CSO & $\surd$ &
\\\hline

Expand synch. before & EXSB & $\surd$ &
\\\hline

Expand synch. after & EXSA & $\surd$ &
\\\hline

Remove synch. around synch. & RSAS & $\surd$ & $\surd$
\\\hline

Remove synch. around variable & RSAV & $\surd$ & $\surd$
\\\hline

Remove synch. around method & RSM & $\surd$ & $\surd$
\\\hline

Remove synch. around block & RSB & $\surd$ & $\surd$
\\\hline

Shrink synch. before & SHSB & & $\surd$
\\\hline

Shrink synch. after & SHSA & & $\surd$
\\\hline

\end{tabular}
\caption{Operators used by ARC in the functional and non-functional phases respectively.}
\label{tbl:operators}
\end{center}
\end{table}

\begin{table}
\begin{center}
\begin{tabular}{l l}

\vspace{2mm}

\begin{minipage}{3.70cm}
\footnotesize{\textbf{ Program $P$:}}
\begin{lstlisting}[language=Java]
  ...
  obj.write(var1);
  synchronized(lock){
    myHash.remove(var1);
  }
\end{lstlisting}
\end{minipage}\hfill

&

\begin{minipage}{3.70cm}
\footnotesize{\textbf{ Program $P'$:}}
\begin{lstlisting}[language=Java]
  ...
  synchronized(lock){
    obj.write(var1);
    myHash.remove(var1);
  }
\end{lstlisting}
\end{minipage}

\end{tabular}
\caption{Example of EXSB, the expand synchronization before operator.}
\label{tbl:EXSBExample}
\end{center}
\end{table}

Evolutionary strategies are solely mutation driven. During the fixing phase ARC adds, expands and alters synchronization blocks in an attept to fix deadlocks and data races. During the second phase ARC attempts to shrink and remove synchronization to improve performance. Table~\ref{tbl:operators} lists the operators used in ARC by phase. 

Mutation is a multi-step process. Each member of the ES population contains it's
own copy of the Java project to be fixed.  During the mutation phase each TXL operator operates against each source code unit to create all mutants of it's type. For example, the expand synchronization before (EXSB, Table~\ref{tbl:EXSBExample}) operator is applied to every existing concurrency block in every Java file. All mutants - where each concurrency block is expanded at the front by one statement - are exaustively written to separate files. This is repeated for every Java file in the project and each elidgeable mutant in phase.

With all mutants written to files, an operator is randomly selected to become the actual mutation operator for that member project for that generation.  One of it's mutations is randomly selected and applied to the project as the actual mutation.

Note that each member project must have it's mutants tracked separately as the application of mutations causes the points where future mutation can happen to diverge.  (If a synchronization block is added to the first project, EXSB, EXSA, SHSB and SHSA can operate on it in future generations. The second project may expand an existing synchronization and so-on.)

Statistics are gathered on the performance of each operator.  After a defineable number of generations a weighting scheme is applied:  Mutation operators that were more successful in the past are given higher weights so they are more likely to be selected in the current generation.  Weightings are designed to insure the chance of selection is always greater than zero.  Further, a sliding window is used to dynamically update the weights as the evolution progresses.

As ARC only adds, removes and changes synchronization blocks, the meaning of a program shouldn't change. 


\subsection{Fitness Functions}
\label{sec:funcfitnessfunctions}

A key problem in parallel programming is the unpredictability of thread interleavings.  If a parallel bug appears only rarely, how can we gain confidence that a proposed fix actually works?  IBM's CONTest Tool (cite) is used to instrument the project and inject noise into the scheduled interleavings.  CONTest is then invoked by ARC many times to test each member program in turn.  By running CONTest multiple times we gain confidence enough interleavings are explored to find bugs. Note that running CONTest many times for each member of the population at each generation is time intensive.

In the first phase fitness is determined by the number of successfull CONTest runs. As a data race could also be a successful execution, they also add to fitness. The weight for data races, $WDataRace$ is set lower than $WSuccess$ to reflect this. 

\begin{center}
$Fitness = \# Successes x WSuccess + \# Data Races * WDataRace$ 
\end{center}

Phase one ends when a ES member is found that passes all CONTest runs. There is a possibility that the bug isn't truly fixed in the proposed solution.  By chance the proper interleavings were not explored.  ARC has an additional safeguard for this:  Every proposed fix is run a large number of times in CONTest. (The default is 10x more runs than a ES member experiences in any generation.) If any test fails the solution is rejected and the evolutionary process continues.

Once a fix is found the second phase begins.  Here the fitness function is concerned with optimizing performance.  Fitness is now determined by the wall time required for an execution and the number of voluntary context switches made.  Durning the initialization of phase 2 the proposed fix is run a large number of times to determine baseline averages for these two values.

We reason these two metrics are important as excessive synchronization causes contention within a program.  Phase two attempts to reduce contention by removing and reducing synchronization.

Removing and reducing synchronization runs the risk of introducing new bugs in to the program.  Thus, before the phase 2 fitness evaluation is performed, a phase 1 fitness evaluation is performed again.  If any deadlock or data race is encountered the proposed fix is reset to it's state in the previous generation. 

Assuming correctness hasn't suffered the member program is run again a number of times in CONTest.  Average wall time and voluntary context switches are computed and compared to the baseline values.  Improvements are rewarded with higher fitness.  

As it isn't clear when a member program is as optimized as it could be - and there is no a-priori way to know this, the second phase runs for it's full allotment of generations.  At the end it outputs the member program with the highest non-functional fitness as the final optimized fix.
 
\section{Experiments}




\subsection{Experimental Setup}




\subsection{Experimental Results}




\section{Threats to Validity}




\section{Future Work}




\section{Conclusion}



% Samples we can reuse
\begin{enumerate}
	\item \textit{Synchronize an unprotected shared resource.} One cause of a data race is that a shared resource is unprotected. By synchronizing around a shared resource data races can be fixed.

\vspace{2mm}
	\begin{minipage}{3.70cm}
	\footnotesize{\textbf{ Program $P$:}}
\begin{lstlisting}[language=Java]
...
obj.write( var1 );
...

\end{lstlisting}
\end{minipage}\hfill
\begin{minipage}{3.70cm}
\footnotesize{\textbf{ Program $P'$:}}
\begin{lstlisting}[language=Java]
synchronized ( lock ){
	obj.write( var1 );
}
\end{lstlisting}
\end{minipage}


	\item \textit{Expand synchronization regions to include unprotected source code.} Data races can sometimes be caused if the synchronization region does not fully encapsulate access to the shared resources. Expanding the synchronization region can also fix the data race.

\vspace{2mm}
	\begin{minipage}{3.70cm}
\footnotesize{\textbf{ Program $P$:}}
\begin{lstlisting}[language=Java]
synchronized ( lock ){
	obj.write( var1 );
}
obj.write( var2 );
\end{lstlisting}
\end{minipage}\hfill
\begin{minipage}{3.70cm}
\footnotesize{\textbf{ Program $P'$:}}
\begin{lstlisting}[language=Java]
synchronized ( lock ){
	obj.write( var1 );
	obj.write( var2 );
}
\end{lstlisting}
\end{minipage}


	\item \textit{Interchange nested lock objects.} Common deadlocks occur due to the ordering of lock acquisition. By interchanging nested lock objects common deadlocks can be fixed.

\vspace{2mm}
	\begin{minipage}{3.70cm}
\footnotesize{\textbf{ Program $P$:}}
\begin{lstlisting}[language=Java]
synchronized ( lock1 ){
	synchronized ( lock2 ){
		obj.write( var1 );
	}
}
\end{lstlisting}
\end{minipage}\hfill
\begin{minipage}{3.70cm}
\footnotesize{\textbf{ Program $P'$:}}
\begin{lstlisting}[language=Java]
synchronized ( lock2 ){
	synchronized ( lock1 ){
		obj.write( var1 );
	}
}
\end{lstlisting}
\end{minipage}

\end{enumerate}


%\notesbox{
\begin{footnotesize}
\begin{center}
$fitness(P) = \displaystyle\sum\limits_{i=0}^n\frac{interleavings\ without\ a\ bug}{total\ \#\ of\ interleavings\ tested}$
\end{center}
\end{footnotesize}
\begin{tiny}
\begin{center}
$n = \#\ of\ Test\ Cases$
\end{center}
\end{tiny}
%}


% use section* for acknowledgement
%\section*{Acknowledgment}
%The authors would like to thank the Natural Sciences and Engineering Research Council of Canada (NSERC) for funding this research.


% trigger a \newpage just before the given reference
% number - used to balance the columns on the last page
% adjust value as needed - may need to be readjusted if
% the document is modified later
%\IEEEtriggeratref{8}
% The "triggered" command can be changed if desired:
%\IEEEtriggercmd{\enlargethispage{-5in}}

% references section

% can use a bibliography generated by BibTeX as a .bbl file
% BibTeX documentation can be easily obtained at:
% http://www.ctan.org/tex-archive/biblio/bibtex/contrib/doc/
% The IEEEtran BibTeX style support page is at:
% http://www.michaelshell.org/tex/ieeetran/bibtex/
\bibliographystyle{IEEEtran}
% argument is your BibTeX string definitions and bibliography database(s)
\bibliography{x2012}
%
% <OR> manually copy in the resultant .bbl file
% set second argument of \begin to the number of references
% (used to reserve space for the reference number labels box)

% that's all folks
\end{document}
