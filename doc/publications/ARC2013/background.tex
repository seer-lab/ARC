\section{Background}
\label{sec:background}

\subsection{Concurrency Bugs}
\label{sec:concurrency}

% Concurrency is the act of having multiple threads executing in parallel.
% Concurrent programs are able to exploit multi-core systems where threads are
% distributed across each of the CPUs to increase performance.

%Concurrency introduces a new class of bugs, the \textit{heisenbugs}. Due to concurrent access to shared memory, threads are able to cause \textit{data
%races} and \textit{deadlocks}.

Data races and deadlocks are two of the most common concurrency bugs. A \textbf{data race} can be defined as: \textit{``\ldots two or more concurrent threads access a shared variable and when at least one access is a write, and the threads use no explicit mechanism to prevent the access from being simultaneous.''}~\cite{LSW07}. A \textbf{deadlock} can be defined as: \textit{``\ldots a situation where two or more processes are unable to proceed because each is waiting for one of the others to do something in a deadlock cycle \ldots} For example, a deadlock can occur when one thread in a program holds a lock that another thread desires and vice-versa''~\cite{LSW07}.

Improper synchronization in concurrent Java programs allows data races and deadlocks to occur. 
%ARC attempts to fix these bugs by evolving a program that contains proper synchronization.
These bugs are extremely difficult to detect due to the non-deterministic nature of how threads are interleaved (i.e., the way the system schedules threads). Various techniques are available to detect concurrency bugs such as static analysis~\cite{NA07,NPSG09,HP04}, stress testing~\cite{HSU03}, dynamic analysis~\cite{JNPS09,EFN+02}, and model checking~\cite{BHPV00,RDH03,OM03,MQB07,Holz97,JM04,HP00}.
% TODO Possibly trim out some of the techniques for bug detection?

\subsection{Evolutionary Strategies}
\label{sec:evolutionary_strategies}

ES is an example of a heuristic search algorithm and shares some similarities  with the most commonly referenced heuristic search technique, the genetic algorithm (GA)~\cite{GA92}. Due to space limitations, we will now provide only a brief outline of ES.

A standard GA is population based, uses mutation, crossover and a fitness function to evolve solutions to problems. A proposed solution to the problem is encoded as an individual of the population. Each individual is evaluated by a fitness function, which is an equation that determines how close the individual is to the solution. The more fit a individual's solution is, the greater the chance it will pass it's genetic material (i.e., part of the solution) on to the next generation.
%\textit{Selection pressure preferentially selects better solutions.}
Crossover mixes the individuals to produce new ones while mutation injects fresh information in to the population so it does not become stagnant.

ES is a simpler form of search than a genetic algorithm. A population of proposed solutions is generated and mutated each generation. Crossover and selection aren't used. Every individual's fitness is evaluated every generation. ES end when a predetermined fitness is reached, after a set number of generations have passed or after a search budget is exhausted.

ARC uses an ES instead of a GA because intuitively, crossover does not seem to fit well with the problem. In particular, using crossover to mix two states of a program in relation to repairing concurrency could lead to more troubles in regards to merging the two code bases. Our rational is to use ES and leave crossover/selection as possible future work.
